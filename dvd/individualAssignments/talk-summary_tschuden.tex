\documentclass[a4paper]{article}

%% Language and font encodings
\usepackage[english]{babel}
\usepackage[utf8x]{inputenc}
\usepackage[T1]{fontenc}

%% Sets page size and margins
\usepackage[a4paper,top=2cm,bottom=2cm,left=3cm,right=3cm,marginparwidth=1.75cm]{geometry}

%% Useful packages
\usepackage{amsmath}
\usepackage{graphicx}
\usepackage[colorinlistoftodos]{todonotes}
\usepackage[colorlinks=true, allcolors=blue]{hyperref}

\providecommand{\keywords}[1]{\textbf{\textit{Tags:}} #1}
\providecommand{\talkurl}[1]{\textbf{\textit{Url:}} #1}
\providecommand{\track}[1]{\textbf{\textit{Track:}} #1}
\providecommand{\speaker}[1]{\textbf{\textit{Speaker:}} #1}



\title{Game Discoverability Day: Crowdfunding Your Video Game in 2019 by Thomas Bidaux}
\author{Author: Julia Tschuden}

\begin{document}
\maketitle

\begin{keywords} crowdfunding, video games, raising capital, tips and tricks \end{keywords}

\begin{track} GDD 2019 - Business \& Marketing \end{track}

\begin{talkurl} \url{https://www.gdcvault.com/play/1025708/Game-Discoverability-Day-Crowdfunding-Your} \end{talkurl}

\begin{speaker}Thomas Bidaux, ICO Partners \end{speaker}


\section{Summary of Talk}

In this section the talk of Thomas Bidaux is summarized.

\subsection{State of play}
Right now, kickstarter is the biggest and most popular crowdfunding platform used worldwide. On average one video game campaign per day is started on kickstarter. In 2018 approximately 15 million dollars where achieved through crowdfunding on this website. However, half of those video games raised less than 10.000 dollars during the lifespan of their campaign. Only 60+ video games raised more than 50.000 dollars.

\subsection{What projects are good for crowdfunding?}
All games are good for crowdfunding besides: games for kids, free to play (F2P) games and mobile games. F2P games are bad for crowdfunding because a player usually first tests the game before deciding if its worth spending money on. Mobile games are bad for crowdfunding because most people think that mobile games are inferior to computer or console games.

\subsection{When is a game ready for crowdfunding?}
In early 2012 a game was ready for crowdfunding at the beginning of production. This worked because people wanted to give power to the developers by weakening publishers. However, economics have chanced, which made people change their expectations. Nowadays people want to see what they invest into. In 2019 the perfect timing for crowdfunding is at the end of production when the alpha is released.

\subsection{Best practices}
The most important message the speaker states in his talk is that crowdfunding is all about love. To be successful with your campaign people need to love your project. If there is no love for the game it will fail. In addition supporters need to have the feeling that the proposed game is going to be finished, so a demonstrable gameplay, a prototype or a demo is needed. Another advice is to launch the campaign at the beginning of the month and end it at the beginning of the next one. People are more willing to invest into projects when their paycheck arrives. The best campaign length ranges from 30 to 35 days. On the last days send e-mails and notifications to your followers, most will back you last minute.\\\\
If your campaign makes 20\% of your goal within the first 48h chances are high that the campaign will be successful. If your campaign starts off slow but manages to get 50\% of the goal within half of the campaigns length it will most likely still be successful.\\\\
To increase your chances of a successful campaign connect to your fans. Talk to them about your project, notify them when the kickstarter starts, do streams on twitch to show the progress of the game. In addition you can do advertising on google and fb, however, don't spend too much money on it.

\subsection{Failing a kickstarter}
If you fail your kickstarter by not raising enough money, don't let it demotivate you. Analyse why your project failed, this helps in making your next crowdfunding campaign better.

\section{Overview and Relevance}
%Research on the topic of the talk; overall overview and the relevance of the technologies/techniques; give a short overview on the state of the art of the topic, reference further readings and current developments. 

%Provide a list of further readings, links (websites, papers, talks, articles,...) in the bibliography 

Crowdfunding is a method of raising money through the efforts of a third party. Compared to traditional ways of raising capital, a big benefit of crowdfunding is that the fundraiser can reach a huge group of people at once. Instead of finding a few big investors, the fundraiser only needs many people which invest a smaller amount of money. In addition the fundraising efforts are centralized and the developer can focus on one thing. According to \cite{cfguide} there are three types of fundraising: donation-based, reward-based and equity-based fundraising. In a donation-based setting, investors don't expect a financial return. In a reward-based setting, fundings are exchanged for rewards. F.e. getting the game at a cheaper price. In a equity-based setting the investors become part-owners of the project.

\subsection{Importance of Crowdfunding}
Money is the most important asset for a project. A successful crowdfunding campaign shows bigger investors, that the market already accepted the project. Convincing the investors to support the project is a lot easier if the masses already support you. The author of this article \cite{importance} also point out, that the fundraiser will receive much feedback and ideas for modifications. But the most important aspect of crowdfunding is, that it is easier than traditional ways of raising capital \cite{importance}.

\subsection{State of the art}
According to the speakers report \cite{cavg}, the first half of 2019 was the best semester for crowdfunded games on Kickstarter since 2015. That is because the number of funded projects increased, while fewer teams try to get crowdfunded. In addition the author of \cite{kfg} points out, that crowdfunding a video game is still difficult in 2019. Only 40\%, or in other words 193 out of 485 games, were crowdfunded in the first half of 2019. If this number is compared to the previous semesters a increase of successful fundings can be seen.

\renewcommand{\refname}{\section{References and Further Sources}}
\begin{thebibliography}{1}

\bibitem{importance}
June 2019,
"Importance of Crowdfunding",
http://www.finsmes.com/2019/06/the-importance-of-crowdfunding.html
\bibitem{failing}
Christopher Dring,
"Kickstarter model sets game developers up to fail.",
September 2019,
https://www.gamesindustry.biz/articles/2019-09-26-the-kickstarter-model-sets-game-developers-up-to-fail

\bibitem{cfguide}
https://www.fundable.com/learn/resources/guides/crowdfunding/what-is-crowdfunding,
\bibitem{kfg}
Christine Fisher,
"Kickstarter video game projects up year over year.",
July 2019,
https://www.engadget.com/2019/07/15/kickstarter-video-game-projects-up-year-over-year/?guccounter=1

\bibitem{cavg}
Thomas Bidaux,
"Crowdfunding and video games 2019 mid year update.",
July 2019,
https://icopartners.com/2019/07/crowdfunding-and-video-games-2019-mid-year-update/

\bibitem{characteristics}
Charles Owen-Jackson,
"Characteristics of a successful video game."
July 2018,
https://blog.vanillaforums.com/gaming-5-characteristics-of-a-successful-video-game-crowdfunding-campaign

\bibitem{rewards}
"What are the best rewards for a video game crowdfunding campaign on kickstarter?",
May 2017,
https://www.quora.com/What-are-the-best-rewards-for-a-video-game-crowdfunding-campaign-on-kickstarter


\end{thebibliography}

\end{document}

