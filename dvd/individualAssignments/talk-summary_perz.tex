\documentclass[a4paper]{article}

%% Language and font encodings
\usepackage[english]{babel}
\usepackage[utf8x]{inputenc}
\usepackage[T1]{fontenc}

%% Sets page size and margins
\usepackage[a4paper,top=2cm,bottom=2cm,left=3cm,right=3cm,marginparwidth=1.75cm]{geometry}

%% Useful packages
\usepackage{amsmath}
\usepackage{graphicx}
\usepackage[colorinlistoftodos]{todonotes}
\usepackage[colorlinks=true, allcolors=blue]{hyperref}

\providecommand{\keywords}[1]{\textbf{\textit{Tags:}} #1}
\providecommand{\talkurl}[1]{\textbf{\textit{Url:}} #1}
\providecommand{\track}[1]{\textbf{\textit{Track:}} #1}
\providecommand{\speaker}[1]{\textbf{\textit{Speaker:}} #1}


\title{Procedurally Crafting Manhattan for Marvel by David Santiago}
\author{Author: Markus Perz}

\begin{document}
\maketitle

\begin{keywords}Procedurally generated game content, GDC 2019, Spider-Man\end{keywords}

\begin{track} Game Developers Conference 2019 - Programming ) \end{track}

\begin{talkurl}  \url{https://www.gdcvault.com/play/1025765/Procedurally-Crafting-Manhattan-for-Marvel} \end{talkurl}

\begin{speaker}David Santiago, Insomniac Games\end{speaker}


\begin{abstract}

Game content is getting more attention by players and more effort is required to fit expectations for high quality AAA-titles. To handle this amount of work hybrid procedures of generating parts of the game as well as handcrafting the details, can allow the creation of large open world maps and still consider quality of details. David Santiago demonstrates how Insomniac Games faced the challenge of creating Manhattan from scratch while being on a strict time schedule. Planing, communication and clever dependency management are the essential parts when developing hybrid game worlds that are 80\% procedurally generated and 20\% handcrafted .

\end{abstract}

\section{Summary of Talk}

Principal technical artist David Santiago describes how his team at Insomniac Games faced the challenge of crafting the borough of Manhattan for their AAA-Game 'Marvel's Spider-Man' that was released in September 2018 .
Having developed the title 'Sunset Overdrive' with similar requirements, David Santiago was chosen to be appropriately qualified for this task.
But after some ground work his team soon realised that the generation of an open world Manhattan had to be done in much more sophisticated way than expected to fit the required expectations of the fans while still coping with a relatively close release date. Santiago mentions the pipeline of their generation tool, the balance between procedurally generated and handcrafted content while also giving insights into the lessons learned while designing such a system. The goal was that most of the buildings, all the traffic lines, roads and pavements, as well as some light sources should be placed in the map procedurally. 

\subsection{Determining constraints and dividing work}

The initial step in developing the system was to subdivide the island of Manhattan into a 2d grid. This helped in communication and naming when referring to places on the map that in total spans an area of 6km by 3km squared. It also helped in future steps since polishing generated content with handmade art could be done in a more organized fashion; tile by tile.

Additionally special regions of the Manhattan, like central park and some more important buildings were chosen to be purely handcrafted, since they would not fit to well into the generation pipeline. The generator would be better suited for buildings, traffic and also lightning. 

\subsection{Streamlining the process of procedural content generation}

The initial goal was to quickly setup working ground, so that other developers could implement their parts of the game as quick as possible. Additionally the benefit of having fast and continues progress was that artists could give their feedback which lead to constant improvement to generated content.

The system was setup to generate the entire Map daily for all developers, but manual request could be issued to generate certain tiles or derestricts in a quicker ways. This was useful for developers fiddling around with the properties of the generation system.

\subsection{Buildings and traffic}

Santiago his team started to lay out all buildings according to a blue print of Manhattan. Areas were approximated to fit a more intuitive, rectangle shape. This shapes were then filled with a modular building set using pieces of Sunset Overdrive. Textures for the buildings were generated by multiple different procedural generation tools but required them to stick to similar sized metrics for similar types of buildings. Since Spider-Man's main way of navigating through the map would be by tossing his webs, special detail on roofs and facades was also necessary (See \ref{looknfeel}). 

After generating static regions for the buildings, a sophisticated road system takes empty spaces together with handmade constraints and filled areas with pavements, roads and navigation paths for NPC characters. Since the primary goal in the game was to save the civilisation by using the hero his special abilities, much effort has had gone into making NPC navigation and traffic systems feel naturally and well crafted. Since city traffic by its very nature relies on a rule set, laying road and streets and accompanying assets like street lanterns and traffic lights were very suitable for the job of procedural generation. Nevertheless it is important to get all the dependencies done right, and it often happened that later changes to the game would require adjustments in the constrains used by generation.

\subsection{Balancing generated and handcrafted content}

Procedurally generated content is the clear winner when it comes to speed. Santiago mentions that the idea was to give their artists and designers a head start by procedurally generation large chunks of the city before letting individual artist go into more detail. The goal was to achieve roughly 80\% of the content by generation and spend as much time as needed into polishing the remaining 20\% hand-authored assets.
To achieve this game content had to be classified and separated into blocks that were worth more attention and were potentially more interesting for the player, and objects that could easily be generated without losing much quality of the game.
Gameplay intense areas and building blocks near key-scenes of the game were focused by the artists, but also individual parts of the else generated buildings were declared more important.
Specifically, store fronts and many object on the ground level were handmade, but generic enough to be propped and reused with different materials \cite{gdc2019b}.

\subsection{Fitting the "Look and Feel" of the game}
\label{looknfeel}

A major goal of the team behind Spider-Man was to allow the player to have as much fun while navigating the map as possible, ideally the player would rarely walk but most of the time tend to the much faster and joyful way of movement; web tossing. To achieve the well known navigation characteristics of the hero, the buildings had to contain some anchor points (for the web to land) at good fitting distances that were situated to be accessible mid air. This again could be laid out perfectly in a series of rules that would influence map generation. Lanterns, iconic New York style fire ladders and facades would be placed were needed to enable proper navigation and fitting the style of Spider-Man seamlessly.

\subsection{Lessons learned}

The key lesson Santiago mentions in his talk, is that early understanding of connections between different systems can lead to time savings. A good example is the imposter system. Its sole  purpose was to generate low detailed representation of each building in the city, to be used in representations where the full city of Manhattan is visible. This system relied on basically every other generation step since laying out small representations of the procedurally crafted buildings required the building itself but also the generated streets, connections, textures, decals etc.

While developing the generation system a key task is also to communicate the workflow and ideas with other team members. It is not feasible to generate the newest changes daily if some developers refuse to check their content in daily. Also certain areas of the map, where e.g. different districts crossed, required more authored attention and communication.

As a conclusion, it is wrong to say procedural generated system are setup once and take off the entire workload of developers, since much work needs to go in understanding and leveraging the strengths of the system. 

\section{Overview and Relevance}

While Spider-Man fits naturally in this open world setting, many developer can benefit by in cooperating similar systems into their games. 

\subsection{Benefits of procedural generation}
Since the demand in player experience and art design is constantly rising it is not only challenging for indie developers to come up with new quality content. By leveraging modern computing powers implementing procedural generation for a broad set of assets can take some work of artists and quickly create large game worlds \cite{green_2016}. 

Since procedurally generated content relies on a rule set that describes how parts of the final world are laid out, quick adjustments to these rules can lead to better results across the map. During presentation Santiago emphasized how changes to the traffic system would occur by adjusting different constraints or changing positions of hand-authored content. This again economizes time and money. Many generation system do not require any heavily intelligent algorithms and can be reduced to the outcome of simple random number generation. Often players tend to interpret more complexity into easy developed systems \cite{short2019}.

Procedurally generated games can also enhance re-playability. Laying out the world procedurally can change scenes of the game dynamically and prevent players from losing interest \cite{green_2016}. Spider-Man could really profit from this, since up to  3000 different crimes were placed over the map and NPC interaction as well as the player input mandate how each instance would take place. 


\subsection{Other games that rely on procedural generated content}

Many games leverage the strength of procedural generated game objects and maps. A very detailed list can be found on Wikipedia \cite{wiki_list}. 

One of the more prominent games on this list is Minecraft, which makes heavy use of procedurally generated game tiles to build the map. Without such a system it would not be possible to create a map that is made of more than two hundred quadrillion blocks \cite{minecraft}.

A very prominent title which relied on procedural generated content is Hello Games' "No Man's Sky" which was released for PlayStation 4 and Windows in August 2016. The game itself featured a large universe with more than 18 million planets \cite{nomanssky}. Additionally the idea of the game was to let each planet host its own set of creatures that should also diverge in looks and behaviour. To make all this possible mutation algorithms and generation of creatures based on primitive predefined shapes was vital \cite{nomanssky}.

\subsection{Conclusion}

Procedural generation system will further lead the way in aspects of content generation on a large scale. On top of that it is important to note how handcrafted a game can look, if details are placed properly and the generation pipeline allows hand-authored changes without overwriting them. Making use of the best from both sides player can further expect new original content in potentially unseen dimensions.  

\newpage
\renewcommand{\refname}{\section{References and Further Sources}}

\begin{thebibliography}{1}

\bibitem{gdc2019}
David Santiago,
\emph{Marvel's Spider-Man: Procedurally Crafting Manhattan for Marvel},
Game Developer Conference 2019, San Francisco California,
2019.

\bibitem{gdc2019b}
Matt McAuliffe and Brian Mullen and Ryan Benno,
\emph{Marvel's Spider-Man: A Deep Dive Into the Look Creation of Manhattan},
Game Developer Conference 2019, San Francisco California,
2019.

\bibitem{green_2016}
Green, Dale,
\emph{Procedural content generation for C++ game development},
Packt Pub., Birmingham, 2016.

\bibitem{short2019}
Tanya X. Short,
\emph{Procedural Storytelling in Game Design},
Excerpt at \url{https://www.gamasutra.com/view/news/340190/How_to_effectively_use_procedural_generation_in_games.php}
CRC Press, 2019.


\bibitem{wiki_list}
Wikipedia,
\emph{List of games using procedural generation},
\url{https://en.wikipedia.org/wiki/List_of_games_using_procedural_generation}, Wikipedia, 2019

\bibitem{minecraft}
Jeremy Peel,
\emph{Just how big is a Minecraft world? Big, as it turns out},
\url{https://www.pcgamesn.com/minecraft/just-how-big-minecraft-world-big-it-turns-out}, PCGamesN, 2012

\bibitem{nomanssky}
Rambus Press,
\emph{The algorithms of No Man’s Sky},
\url{https://www.rambus.com/blogs/the-algorithms-of-no-mans-sky-2/}, Rambus, 2016



\end{thebibliography}

Note: To enhance readability all citations to \cite{gdc2019} were left out intentionally, since the goal of this work was to give a quick overview of this talk as well as further readings, which would result in repeating the same citation constantly. It should be clear that this was used as the main reference.

\end{document}
